%packages%
\usepackage[ngerman]{babel}
\usepackage{cite}
%\usepackage[numbers,square]{natbib}
\usepackage[utf8x]{inputenc}
%\usepackage[svgnames]{xcolor}
\usepackage[dvipsnames]{xcolor} 
\usepackage{comment}
\usepackage{fancyhdr}
\usepackage{fancyvrb}
\usepackage{listings}
\usepackage{graphicx}
\usepackage{nomencl}
\usepackage{float}
\usepackage{lastpage}
\usepackage{caption}
\usepackage{tikz}
\usepackage{wrapfig}
\usepackage{tipa}
\usepackage{lmodern}
%\usepackage[bookmarks=true,colorlinks=true,pagecolor=black,linktocpage=true]{hyperref}
\usepackage{amsmath,amsfonts,amsthm}
\usepackage[ruled,vlined]{algorithm2e}
%\usepackage[lined,boxed,commentsnumbered]{algorithm2e}
\usepackage{setspace}
\usepackage{geometry}
%\usepackage{svgnames}{xcolor}
%\PassOptionsToPackage{xcolor=dvipsnames}{beamer}
%\usepackage[svgnames]{xcolor}
%\usepackage{xcolor}
\usepackage{tabularx}

\usepackage[
    bookmarks,
    bookmarksopen=true,
    colorlinks=true,
    linkcolor=red, 
    citecolor=OliveGreen,
    urlcolor=blue,  
%    linkcolor=black, 
%    citecolor=black,
%    urlcolor=black,     
    anchorcolor=black, 
    filecolor=black,
    menucolor=red, 
    backref,
    plainpages=false, 
    pdfpagelabels,
    hypertexnames=false, 
    linktocpage 
]{hyperref}

\usepackage[
	nohyperlinks,
	printonlyused
]{acronym}

\usepackage{bibtopic}

% für source code
\definecolor{mygreen}{rgb}{0,0.6,0}
\definecolor{mygray}{rgb}{0.5,0.5,0.5}
\definecolor{myred}{rgb}{1,0,0}
\definecolor{myblue}{rgb}{0,0,1}

\lstset{ %
  %backgroundcolor=\color{white},   % choose the background color; you must add \usepackage{color} or \usepackage{xcolor}
  basicstyle=\ttfamily\footnotesize,        % the size of the fonts that are used for the code
  breakatwhitespace=false,         % sets if automatic breaks should only happen at whitespace
  breaklines=true,                 % sets automatic line breaking
  captionpos=b,                    % sets the caption-position to bottom
  commentstyle=\color{mygreen},    % comment style
  deletekeywords={...},            % if you want to delete keywords from the given language
  escapeinside={\%*}{*)},          % if you want to add LaTeX within your code
  extendedchars=true,              % lets you use non-ASCII characters; for 8-bits encodings only, does not work with UTF-8
  keepspaces=true,                 % keeps spaces in text, useful for keeping indentation of code (possibly needs columns=flexible)
  keywordstyle=\color{blue},       % keyword style
  frame=bt,
  language=XML,                 	% the language of the code
  numbers=left,                    % where to put the line-numbers; possible values are (none, left, right)
  showspaces=false,                % show spaces everywhere adding particular underscores; it overrides 'showstringspaces'
  showstringspaces=false,          % underline spaces within strings only
  showtabs=false,                  % show tabs within strings adding particular underscores
  stepnumber=1,                    % the step between two line-numbers. If it's 1, each line will be numbered
  stringstyle=\color{myblue},     % string literal style
  tabsize=2,                       % sets default tabsize to 2 spaces  
  title=\lstname                   % show the filename of files included with \lstinputlisting; also try caption instead of title
}

%% obere reihe wörter die vorne stehen, also auseinander 
%% untere reihe wörter die innen stehen, also zusammen
%\lstdefinestyle{xmlandroid} {
% morekeywords={
%  encoding, service, name, class,
%  android:name, android:minSdkVersion, android:targetSdkVersion, android:protectionLevel, android:value, android:allowBackup, android:icon, android:label, android:theme, xmlns:android, android:id, xmlns:tools,android:layout_width, android:layout_height, android:layout_gravity, android:hint, android:ems, android:text, android:onClick, android:orientation}
% }
 

%\lstdefinestyle{xmltomcat} {
%  morekeywords={encoding, web, app, servlet, name, class, mapping, url, pattern}
%}

%\lstdefinelanguage{XML}
%{
%  basicstyle=\ttfamily\footnotesize,
%  moredelim=[s][\color{red}]{\ }{=},
%  moredelim=[s][\color{Maroon}]{<}{>},
%  moredelim=[s][\color{Maroon}]{</}{>},
%  morestring=[s][\color{blue}]{"}{"},
%  morecomment=[s]{<?}{?>},
%  morecomment=[s]{<!--}{-->},
%  commentstyle=\color{DarkOliveGreen},
%  stringstyle=\color{black}
%}

\lstdefinelanguage{XML}
{
  %basicstyle=\ttfamily,
  morestring=[s]{"}{"},
  %morecomment=[s]{?}{?>},
  morecomment=[s]{<!--}{-->},
  commentstyle=\color{DarkOliveGreen},
  %moredelim=[s][\color{black}]{>}{</},
  moredelim=[s][\color{black}]{>}{<},
  moredelim=[s][\color{red}]{\ }{=},
  stringstyle=\color{blue},
  identifierstyle=\color{Maroon},
  keywordstyle=\color{red},
  %morekeywords={xml,version,type}% list your attributes here
}

%\usepackage[font=normalsize,format=plain,labelfont={bf,normalsize},textfont={it,normalsize}]{caption}
%\usepackage{courier}
%
%\definecolor{lightgray}{gray}{0.9}
%\definecolor{gray}{rgb}{0.4,0.4,0.4}
%\definecolor{darkblue}{rgb}{0.0,0.0,0.6}
%\definecolor{cyan}{rgb}{0.0,0.6,0.6}

%\lstset{
%  basicstyle=\footnotesize\ttfamily, % Standardschrift
%  %numbers=left, % Ort der Zeilennummern
%  numberstyle=\tiny, % Stil der Zeilennummern
%  %stepnumber=2, % Abstand zwischen den Zeilennummern
%  numbersep=5pt, % Abstand der Nummern zum Text
%  tabsize=2, % Groesse von Tabs
%  extendedchars=true, %
%  breaklines=true, % Zeilen werden Umgebrochen
%  keywordstyle=\color{red},
%    frame=b,
%  % keywordstyle=[1]\textbf, % Stil der Keywords
%  % keywordstyle=[2]\textbf, %
%  % keywordstyle=[3]\textbf, %
%  % keywordstyle=[4]\textbf, \sqrt{\sqrt{}} %
%  stringstyle=\color{white}\ttfamily, % Farbe der String
%  showspaces=false, % Leerzeichen anzeigen ?
%  showtabs=false, % Tabs anzeigen ?
%  xleftmargin=17pt,
%  framexleftmargin=17pt,
%  framexrightmargin=5pt,
%  framexbottommargin=4pt,
%  %backgroundcolor=\color{lightgray},
%  showstringspaces=false % Leerzeichen in Strings anzeigen ?
%}

%\lstdefinelanguage{XML}
%{
%  basicstyle=\ttfamily,
%  morestring=[b]",
%  morestring=[s]{>}{<},
%  morecomment=[s]{},
%  stringstyle=\color{blue},
%  identifierstyle=\color{Maroon},
%  keywordstyle=\color{cyan},
%  morekeywords={xmlns,version,type}% list your attributes here
%}

%\DeclareCaptionFont{white}{\color{white}}
%\DeclareCaptionFormat{listing}{\colorbox{gray}{\parbox{\textwidth}{#1#2#3}}}
%\captionsetup[lstlisting]{format=listing,labelfont=white,textfont=white}
