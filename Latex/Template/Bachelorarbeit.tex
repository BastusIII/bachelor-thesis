%%% latex configuration start %%%
\documentclass[
							a4paper, 
							11pt, 
							openany, 
							liststotoc,
							parskip=half, 
   							headings=normal
						]{scrreprt}						
% all needed packages %
%packages%
\usepackage[ngerman]{babel}
\usepackage{cite}
%\usepackage[numbers,square]{natbib}
\usepackage[utf8x]{inputenc}
%\usepackage[svgnames]{xcolor}
\usepackage[dvipsnames]{xcolor} 
\usepackage{comment}
\usepackage{fancyhdr}
\usepackage{fancyvrb}
\usepackage{listings}
\usepackage{graphicx}
\usepackage{nomencl}
\usepackage{float}
\usepackage{lastpage}
\usepackage{caption}
\usepackage{tikz}
\usepackage{wrapfig}
\usepackage{tipa}
\usepackage{lmodern}
%\usepackage[bookmarks=true,colorlinks=true,pagecolor=black,linktocpage=true]{hyperref}
\usepackage{amsmath,amsfonts,amsthm}
\usepackage[ruled,vlined]{algorithm2e}
%\usepackage[lined,boxed,commentsnumbered]{algorithm2e}
\usepackage{setspace}
\usepackage{geometry}
%\usepackage{svgnames}{xcolor}
%\PassOptionsToPackage{xcolor=dvipsnames}{beamer}
%\usepackage[svgnames]{xcolor}
%\usepackage{xcolor}
\usepackage{tabularx}

\usepackage[
    bookmarks,
    bookmarksopen=true,
    colorlinks=true,
    linkcolor=red, 
    citecolor=OliveGreen,
    urlcolor=blue,  
%    linkcolor=black, 
%    citecolor=black,
%    urlcolor=black,     
    anchorcolor=black, 
    filecolor=black,
    menucolor=red, 
    backref,
    plainpages=false, 
    pdfpagelabels,
    hypertexnames=false, 
    linktocpage 
]{hyperref}

\usepackage[
	nohyperlinks,
	printonlyused
]{acronym}

\usepackage{bibtopic}

% für source code
\definecolor{mygreen}{rgb}{0,0.6,0}
\definecolor{mygray}{rgb}{0.5,0.5,0.5}
\definecolor{myred}{rgb}{1,0,0}
\definecolor{myblue}{rgb}{0,0,1}

\lstset{ %
  %backgroundcolor=\color{white},   % choose the background color; you must add \usepackage{color} or \usepackage{xcolor}
  basicstyle=\ttfamily\footnotesize,        % the size of the fonts that are used for the code
  breakatwhitespace=false,         % sets if automatic breaks should only happen at whitespace
  breaklines=true,                 % sets automatic line breaking
  captionpos=b,                    % sets the caption-position to bottom
  commentstyle=\color{mygreen},    % comment style
  deletekeywords={...},            % if you want to delete keywords from the given language
  escapeinside={\%*}{*)},          % if you want to add LaTeX within your code
  extendedchars=true,              % lets you use non-ASCII characters; for 8-bits encodings only, does not work with UTF-8
  keepspaces=true,                 % keeps spaces in text, useful for keeping indentation of code (possibly needs columns=flexible)
  keywordstyle=\color{blue},       % keyword style
  frame=bt,
  language=XML,                 	% the language of the code
  numbers=left,                    % where to put the line-numbers; possible values are (none, left, right)
  showspaces=false,                % show spaces everywhere adding particular underscores; it overrides 'showstringspaces'
  showstringspaces=false,          % underline spaces within strings only
  showtabs=false,                  % show tabs within strings adding particular underscores
  stepnumber=1,                    % the step between two line-numbers. If it's 1, each line will be numbered
  stringstyle=\color{myblue},     % string literal style
  tabsize=2,                       % sets default tabsize to 2 spaces  
  title=\lstname                   % show the filename of files included with \lstinputlisting; also try caption instead of title
}

%% obere reihe wörter die vorne stehen, also auseinander 
%% untere reihe wörter die innen stehen, also zusammen
%\lstdefinestyle{xmlandroid} {
% morekeywords={
%  encoding, service, name, class,
%  android:name, android:minSdkVersion, android:targetSdkVersion, android:protectionLevel, android:value, android:allowBackup, android:icon, android:label, android:theme, xmlns:android, android:id, xmlns:tools,android:layout_width, android:layout_height, android:layout_gravity, android:hint, android:ems, android:text, android:onClick, android:orientation}
% }
 

%\lstdefinestyle{xmltomcat} {
%  morekeywords={encoding, web, app, servlet, name, class, mapping, url, pattern}
%}

%\lstdefinelanguage{XML}
%{
%  basicstyle=\ttfamily\footnotesize,
%  moredelim=[s][\color{red}]{\ }{=},
%  moredelim=[s][\color{Maroon}]{<}{>},
%  moredelim=[s][\color{Maroon}]{</}{>},
%  morestring=[s][\color{blue}]{"}{"},
%  morecomment=[s]{<?}{?>},
%  morecomment=[s]{<!--}{-->},
%  commentstyle=\color{DarkOliveGreen},
%  stringstyle=\color{black}
%}

\lstdefinelanguage{XML}
{
  %basicstyle=\ttfamily,
  morestring=[s]{"}{"},
  %morecomment=[s]{?}{?>},
  morecomment=[s]{<!--}{-->},
  commentstyle=\color{DarkOliveGreen},
  %moredelim=[s][\color{black}]{>}{</},
  moredelim=[s][\color{black}]{>}{<},
  moredelim=[s][\color{red}]{\ }{=},
  stringstyle=\color{blue},
  identifierstyle=\color{Maroon},
  keywordstyle=\color{red},
  %morekeywords={xml,version,type}% list your attributes here
}

%\usepackage[font=normalsize,format=plain,labelfont={bf,normalsize},textfont={it,normalsize}]{caption}
%\usepackage{courier}
%
%\definecolor{lightgray}{gray}{0.9}
%\definecolor{gray}{rgb}{0.4,0.4,0.4}
%\definecolor{darkblue}{rgb}{0.0,0.0,0.6}
%\definecolor{cyan}{rgb}{0.0,0.6,0.6}

%\lstset{
%  basicstyle=\footnotesize\ttfamily, % Standardschrift
%  %numbers=left, % Ort der Zeilennummern
%  numberstyle=\tiny, % Stil der Zeilennummern
%  %stepnumber=2, % Abstand zwischen den Zeilennummern
%  numbersep=5pt, % Abstand der Nummern zum Text
%  tabsize=2, % Groesse von Tabs
%  extendedchars=true, %
%  breaklines=true, % Zeilen werden Umgebrochen
%  keywordstyle=\color{red},
%    frame=b,
%  % keywordstyle=[1]\textbf, % Stil der Keywords
%  % keywordstyle=[2]\textbf, %
%  % keywordstyle=[3]\textbf, %
%  % keywordstyle=[4]\textbf, \sqrt{\sqrt{}} %
%  stringstyle=\color{white}\ttfamily, % Farbe der String
%  showspaces=false, % Leerzeichen anzeigen ?
%  showtabs=false, % Tabs anzeigen ?
%  xleftmargin=17pt,
%  framexleftmargin=17pt,
%  framexrightmargin=5pt,
%  framexbottommargin=4pt,
%  %backgroundcolor=\color{lightgray},
%  showstringspaces=false % Leerzeichen in Strings anzeigen ?
%}

%\lstdefinelanguage{XML}
%{
%  basicstyle=\ttfamily,
%  morestring=[b]",
%  morestring=[s]{>}{<},
%  morecomment=[s]{},
%  stringstyle=\color{blue},
%  identifierstyle=\color{Maroon},
%  keywordstyle=\color{cyan},
%  morekeywords={xmlns,version,type}% list your attributes here
%}

%\DeclareCaptionFont{white}{\color{white}}
%\DeclareCaptionFormat{listing}{\colorbox{gray}{\parbox{\textwidth}{#1#2#3}}}
%\captionsetup[lstlisting]{format=listing,labelfont=white,textfont=white}

% all needed commands %
% Commands %
\newcommand{\ua}{\mbox{u.\,a.\ }}
\newcommand{\zB}{\mbox{z.\,B.\ }}
\newcommand{\dahe}{\mbox{d.\,h.\ }}
\newcommand{\Vgl}{Vgl.\ }
\newcommand{\bzw}{bzw.\ }
\newcommand{\evtl}{evtl.\ }

%%% latex configuration end %%%

\begin{document}

%%% document configuration start %%%

% headline and footline %
%Kopf- und Fußzeile
\pagestyle{fancy}
\fancyhf{}
 
%Kopfzeile mittig mit Kaptilname
\fancyhead[L]{\textsf{\nouppercase{\leftmark}}}

%Linie oben
\renewcommand{\headrulewidth}{0.5pt}
	\fancyfoot[R]{\thepage}
%Linie unten
\renewcommand{\footrulewidth}{0.5pt}
 
% Fußzeile auf jeder Seite - auch Kapitel und Inhaltsverzeichnis
\fancypagestyle{plain}{%
   \fancyhf{}%
	\fancyfoot[R]{\thepage}
   \renewcommand{\headrulewidth}{0.0pt} %obere Linie ausblenden
}

\addtolength{\headheight}{\baselineskip}
\addtolength{\headheight}{0.61pt}
\addtolength{\footskip}{10pt}
\renewcommand{\headrulewidth}{1pt}% Trennlinie
% title page%
\graphicspath{{pictures/}}
\begin{titlepage}	
		
	\begin{minipage}[b]{1.0\textwidth}
		\centering
		\includegraphics[width=0.5\textwidth]{hm_logo_svg.pdf}
		\vspace*{0.6cm}
	\end{minipage}
	
	\begin{minipage}{1.0\textwidth} 
		\centering	
		{\Large Hochschule München}\\[0.5cm]
		{\Large Fakultät für Informatik und Mathematik}\\[1.0cm]
		\textsc{\sffamily \LARGE \bfseries Bachelorarbeit}\\[1.0cm]
		{\sffamily \LARGE \bfseries Entwicklung eines Kartenspiel Prototypen mit Testumfeld für das Spiel Schafkopf}\\[1.0cm]	
		{\sffamily \LARGE Development of a card game prototype and test environment for the game "'Schafkopf"'}\\[2.0cm]	
		{\Large Sebastian Stumpf}\\[0.5cm]
		{\Large 26.04.2014}\\[2cm]
		{\Large Betreut durch:}\\[0.5cm]
		{\Large Prof. Dr. Martin Leitner,}\\[0.5cm]
		{\Large Prof. Dr. Reinhard Schiedermeier}\\[0.5cm]
	\end{minipage}
	
\end{titlepage}

% Definition bis zu welcher Tiefe Überschriften numeriert werden
\setcounter{secnumdepth}{3}
\renewcommand{\arraystretch}{2}

%%% document configuration start %%%

%%%%%%%%%%%%%%%%%%%%%%%%%%%%%%%%%%%%%%%%%%%%%%%%%%%%%%%%%%%
%%% ZUSAMMENFASSUNG
%%%%%%%%%%%%%%%%%%%%%%%%%%%%%%%%%%%%%%%%%%%%%%%%%%%%%%%%%%%
\begin{abstract}
% Ein paar Sätze zur Arbeit
\section*{Kurzzusammenfassung} \label{se:zusammenfassung}
Lorem Ipsum
Lorem Ipsum
\end{abstract}
%%%%%%%%%%%%%%%%%%%%%%%%%%%%%%%%%%%%%%%%%%%%%%%%%%%%%%%%%%%
%%% ZUSAMMENFASSUNG ENDE
%%%%%%%%%%%%%%%%%%%%%%%%%%%%%%%%%%%%%%%%%%%%%%%%%%%%%%%%%%%

% Inhaltsverzeichnis usw.
\pagenumbering{roman}
\tableofcontents 
\clearpage
\pagenumbering{arabic} 
\setcounter{page}{1}

%%%%%%%%%%%%%%%%%%%%%%%%%%%%%%%%%%%%%%%%%%%%%%%%%%%%%%%%%%%
%%% Latex Konstrukte
%%%%%%%%%%%%%%%%%%%%%%%%%%%%%%%%%%%%%%%%%%%%%%%%%%%%%%%%%%%
\chapter{Latex Konstrukte} \label{ch:konstrukte}
\section{Aufzählung} \label{se:konstrukte_aufzaehlung}
\begin{itemize}
	\item item 1
	\begin{itemize}
		\item sub item 11
		\item sub item 12
		\item sub item 13
	\end{itemize}
	\item item 2
	\item item 3
\end{itemize}
\clearpage

\section{Grafiken} \label{se:konstrukte_grafiken}
\begin{figure}[H]
\begin{center}
    \includegraphics[width=1.0\textwidth]{./pictures/hm_logo_svg.pdf}
	\caption[Kurzeintrag Verzeichnis Beispielbild]{Text unter Beispielbild} \label{fig:beispielbild}
\end{center}
\end{figure}

\begin{wrapfigure}[13]{r}{0.6\textwidth}
  \begin{center}
    \includegraphics[width=0.5\textwidth]{./pictures/hm_logo_svg.pdf}
  \end{center}
  \caption[Kurzeintrag Verzeichnis Beispielbild wrapped]{Text unter Beispielbild wrapped} \label{fig:beispielbild_wrapped} 
\end{wrapfigure}
Lorem ipsum dolor sit amet, consetetur sadipscing elitr, sed diam nonumy eirmod tempor invidunt ut labore et dolore magna aliquyam erat, sed diam voluptua. At vero eos et accusam et justo duo dolores et ea rebum. Stet clita kasd gubergren, no sea takimata sanctus est Lorem ipsum dolor sit amet. Lorem ipsum dolor sit amet, consetetur sadipscing elitr, sed diam nonumy eirmod tempor invidunt ut labore et dolore magna aliquyam erat, sed diam voluptua. At vero eos et accusam et justo duo dolores et ea rebum. Stet clita kasd gubergren, no sea takimata sanctus est Lorem ipsum dolor sit amet.
Lorem ipsum dolor sit amet, consetetur sadipscing elitr, sed diam nonumy eirmod tempor invidunt ut labore et dolore magna aliquyam erat, sed diam voluptua. At vero eos et accusam et justo duo dolores et ea rebum. Stet clita kasd gubergren, no sea takimata sanctus est Lorem ipsum dolor sit amet. Lorem ipsum dolor sit amet, consetetur sadipscing elitr, sed diam nonumy eirmod tempor invidunt ut labore et dolore magna aliquyam erat, sed diam voluptua. At vero eos et accusam et justo duo dolores et ea rebum. Stet clita kasd gubergren, no sea takimata sanctus est Lorem ipsum dolor sit amet.
\clearpage

\section{Links} \label{se:konstrukte_links}
In \autoref{ch:konstrukte} werden Latex Konstrukte vermittelt.
\newline
In \autoref{se:links} (Links) werden Latex Links erklärt.
\newline
Ein \ref{fig:beispielbild_wrapped} (Grafik) kurzer Grafik Link.
\clearpage


\section{Abkürzungen} \label{se:konstrukte_abkuerzungen}
\acf{abk} erstes Vorkommen.
\newline
\acs{abk} weitere Vorkommen.
\clearpage

\section{Textgestaltung} \label{se:konstrukte_textgestaltung}
\textbf{big}\\
\textit{\glqq italic\grqq}\\
\texttt{monospace}\\
Trenn\-zeichen definieren\\
\clearpage

\section{Zitate} \label{se:konstrukte_zitate}
\textit{\glqq Dies ist ein Buchzitat mit Seitenzahl.\grqq}\cite[S.92ff.]{bsp:buchzitat}
\newline
\textit{\glqq Dies ist ein Artikelzitat.\grqq}\cite{bsp:artikelzitat}
\newline
\textit{\glqq Dies ist ein Weblinkzitat.\grqq}\cite{bsp:weblinkzitat}
\clearpage

\section{Tabellen} \label{se:konstrukte_tabellen}
\begin{table}[h]\footnotesize
\begin{center}
\begin{tabular}{ | p{3cm} | p{5cm} | p{5cm} |}
  \hline
    \textbf{Überschrift 1} & \textbf{Überschrift 2} & \textbf{Überschrift 3} \\
  \hline
    Inhalt 11 & Inhalt 12 & Inhalt 13 \\
  \hline
    Inhalt 21 & Inhalt 22 & Inhalt 23 \\
  \hline
    Inhalt 31 & Inhalt 32 & Inhalt 33 \\
  \hline
\end{tabular}
\caption[Kurzeintrag Verzeichnis Beispiel Tabelle]{Text unter Beispiel Tabelle} \label{tab:Beispieltabelle}
\end{center}
\end{table}
\clearpage

\section{Listings - Codebeispiele} \label{se:konstrukte_listings}
\subsection{Java}
\begin{lstlisting}[{language=JAVA, label=lst:Java - Codebeispiel, caption={[Kurzeintrag Verzeichnis Java Codebeispiel]{Text unter Java Codebeispiel}}}]
public class MainClass {
  public static void main(String args[]) {
    double a = 3.0, b = 4.0;

    // c is dynamically initialized
    double c = Math.sqrt(a * a + b * b);

    System.out.println("Hypotenuse is " + c);
  }
}
\end{lstlisting}
\clearpage

\subsection{XML}
\begin{lstlisting}[{style=xmlandroid, label=lst:Java - Codebeispiel, caption={[Kurzeintrag Verzeichnis XML Codebeispiel]{Text unter XML Codebeispiel}}}]
<LinearLayout xmlns:android="http://schemas.android.com/apk/res/android"
	android:orientation="vertical" >
	...
	<TextView
		...
		android:id="@+id/tv_testTextView"
		android:text="@string/tv_testTextView" />

	<Button
		...
		android:id="@+id/bt_testButton"
		android:onClick="buttonPressed"
		android:text="@string/bt_testButton" />        
</LinearLayout>
\end{lstlisting}
\clearpage

%%%%%%%%%%%%%%%%%%%%%%%%%%%%%%%%%%%%%%%%%%%%%%%%%%%%%%%%%%%
%%% Latex Konstrukte ENDE
%%%%%%%%%%%%%%%%%%%%%%%%%%%%%%%%%%%%%%%%%%%%%%%%%%%%%%%%%%%
	
%%%%%%%%%%%%%%%%%%%%%%%%%%%%%%%%%%%%%%%%%%%%%%%%%%%%%%%%%%%
%%% EINLEITUNG
%%%%%%%%%%%%%%%%%%%%%%%%%%%%%%%%%%%%%%%%%%%%%%%%%%%%%%%%%%%
\chapter{Einleitung} \label{ch:einleitung}

% Warum habe ich das Thema der Bachelorarbeit gewählt, wie ist der aktuelle Stand
\section{Motivation und Ausgangssituation} \label{se:einleitung_motivation}
Lorem ipsum dolor sit amet, consetetur sadipscing elitr, sed diam nonumy eirmod tempor invidunt ut labore et dolore magna aliquyam erat, sed diam voluptua. At vero eos et accusam et justo duo dolores et ea rebum.
\clearpage

% Wo liegen Probleme, Schwerpunkte bei der Umsetzung
\section{Problembeschreibung} \label{se:einleitung_problem}
Lorem ipsum dolor sit amet, consetetur sadipscing elitr, sed diam nonumy eirmod tempor invidunt ut labore et dolore magna aliquyam erat, sed diam voluptua. At vero eos et accusam et justo duo dolores et ea rebum.
\clearpage

%% Was ist das Ziel der Arbeit, soll am Ende herauskommen
\section{Ziel der Bachelorarbeit} \label{se:einleitung_ziel}
Lorem ipsum dolor sit amet, consetetur sadipscing elitr, sed diam nonumy eirmod tempor invidunt ut labore et dolore magna aliquyam erat, sed diam voluptua. At vero eos et accusam et justo duo dolores et ea rebum.
\clearpage

%% Wie ist die Arbeit aufgebaut und warum habe ich diese Struktur gewählt
\section{Vorgehensweise und Aufbau der Arbeit} \label{se:einleitung_aufbau}
Lorem ipsum dolor sit amet, consetetur sadipscing elitr, sed diam nonumy eirmod tempor invidunt ut labore et dolore magna aliquyam erat, sed diam voluptua. At vero eos et accusam et justo duo dolores et ea rebum.
\clearpage

%% Wie ist die Arbeit formatiert, wie sind Zitate etc. umgesetzt
\section{Hinweise zu Formatierung und Struktur} \label{se:einleitung_struktur}
Lorem ipsum dolor sit amet, consetetur sadipscing elitr, sed diam nonumy eirmod tempor invidunt ut labore et dolore magna aliquyam erat, sed diam voluptua. At vero eos et accusam et justo duo dolores et ea rebum.
\clearpage

%%%%%%%%%%%%%%%%%%%%%%%%%%%%%%%%%%%%%%%%%%%%%%%%%%%%%%%%%%%
%%% EINLEITUNG ENDE
%%%%%%%%%%%%%%%%%%%%%%%%%%%%%%%%%%%%%%%%%%%%%%%%%%%%%%%%%%%

%%%%%%%%%%%%%%%%%%%%%%%%%%%%%%%%%%%%%%%%%%%%%%%%%%%%%%%%%%%
%%% GRUNDLAGEN
%%%%%%%%%%%%%%%%%%%%%%%%%%%%%%%%%%%%%%%%%%%%%%%%%%%%%%%%%%%

% Welche mathematischen Grundlagen werden benötigt, kurze Erklärung, Definition verwendeter Sätze / Regeln
\chapter{Technische Grundlagen} \label{ch:grundlagen}
Lorem ipsum dolor sit amet, consetetur sadipscing elitr, sed diam nonumy eirmod tempor invidunt ut labore et dolore magna aliquyam erat, sed diam voluptua. At vero eos et accusam et justo duo dolores et ea rebum.
\clearpage

%%%%%%%%%%%%%%%%%%%%%%%%%%%%%%%%%%%%%%%%%%%%%%%%%%%%%%%%%%%
%%% GRUNDLAGEN ENDE
%%%%%%%%%%%%%%%%%%%%%%%%%%%%%%%%%%%%%%%%%%%%%%%%%%%%%%%%%%%

%%%%%%%%%%%%%%%%%%%%%%%%%%%%%%%%%%%%%%%%%%%%%%%%%%%%%%%%%%%
%%% UMSETZUNG UND PROGRAMMIERUNG
%%%%%%%%%%%%%%%%%%%%%%%%%%%%%%%%%%%%%%%%%%%%%%%%%%%%%%%%%%%

\chapter{Umsetzung und Programmierung} \label{ch:umsetzung}
Lorem ipsum dolor sit amet, consetetur sadipscing elitr, sed diam nonumy eirmod tempor invidunt ut labore et dolore magna aliquyam erat, sed diam voluptua. At vero eos et accusam et justo duo dolores et ea rebum.
\clearpage

%%%%%%%%%%%%%%%%%%%%%%%%%%%%%%%%%%%%%%%%%%%%%%%%%%%%%%%%%%%
%%% ENDE UMSETZUNG UND PROGRAMMIERUNG
%%%%%%%%%%%%%%%%%%%%%%%%%%%%%%%%%%%%%%%%%%%%%%%%%%%%%%%%%%%

%%%%%%%%%%%%%%%%%%%%%%%%%%%%%%%%%%%%%%%%%%%%%%%%%%%%%%%%%%%
%%% VERZEICHNISSE
%%%%%%%%%%%%%%%%%%%%%%%%%%%%%%%%%%%%%%%%%%%%%%%%%%%%%%%%%%%

% Abbildungsverzeichnis
\listoffigures

% Abkürzungsverzeichnis
\chapter*{Abkürzungsverzeichnis}
\addcontentsline{toc}{chapter}{Abkürzungsverzeichnis} 
\begin{acronym}[Abk.]
\acro{abk} [Abk.] {Abkürzung}
\end{acronym}

% Listingverzeichnis -> Codebeispiele
\lstlistoflistings

% Literaturverzeichnis
\addchap{Literaturverzeichnis}
\bibliographystyle{alphadin}
\begin{btSect}{literatur}
\section*{Literatur}
\btPrintCited
\end{btSect}

\clearpage
% Weblinkverzeichnis
\begin{btSect}{weblinks}
\section*{Weblinks}
\btPrintCited
\end{btSect}

% Tabellenverzeichnis
\listoftables

%%%%%%%%%%%%%%%%%%%%%%%%%%%%%%%%%%%%%%%%%%%%%%%%%%%%%%%%%%%
%%% VERZEICHNISSE ENDE
%%%%%%%%%%%%%%%%%%%%%%%%%%%%%%%%%%%%%%%%%%%%%%%%%%%%%%%%%%%

%%%%%%%%%%%%%%%%%%%%%%%%%%%%%%%%%%%%%%%%%%%%%%%%%%%%%%%%%%%
%%% ANHANG
%%%%%%%%%%%%%%%%%%%%%%%%%%%%%%%%%%%%%%%%%%%%%%%%%%%%%%%%%%%

\renewcommand{\thesection}{\Alph{section}}
\appendix
\addchap{Anhang}
\section{Inhalt der beiliegenden CD}
Inhalte auf der CD. BA im PDF-Format, zitierte Weblinks im PDF Format, die Anwendung, offizielle Spielregeln als PDF, Sonstiges.

%% Erklärung
\newpage
\thispagestyle{empty}
\begin{minipage}[b]{0.5\textwidth} 
Stumpf, Sebastian \\
(Familienname, Vorname)
\end{minipage}
	% Auffüllen des Zwischenraums
	\hfill
	% minipage mit Grafik
\begin{minipage}[b]{0.5\textwidth}
\begin{flushright}
München, 25.04.2014 \\
(Ort, Datum)
\end{flushright}
\end{minipage}
\\\\\\\\
\begin{minipage}[b]{0.5\textwidth} 
25.10.1987 \\
(Geburtsdatum)
\end{minipage}
	% Auffüllen des Zwischenraums
	\hfill
	% minipage mit Grafik
\begin{minipage}[b]{0.5\textwidth}
\begin{flushright}
IF 7W / SS 2014 \\
(Studiengruppe / Semester)
\end{flushright}
\end{minipage}
\\ \\ \\ \\ \\
\subsection*{\centering Erklärung}
%\addcontentsline{toc}{section}{Eidesstattliche Erklärung}%\addtocontents{toc}{\vfill}
Hiermit erkläre ich, dass ich die Bachelorarbeit selbständig verfasst, noch nicht anderweitig für Prüfungszwecke vorgelegt, keine anderen als die angegebenen Quellen oder Hilfsmittel benutzt sowie wörtliche und sinngemäße Zitate als solche gekennzeichnet habe.\\\\\\\\\\
\begin{flushright}
$\overline{~~~~~~~~~~~~~~~~~~~~~~~~~~~~~~~~~~\mbox{(Unterschrift)}}$
\end{flushright}


%%%%%%%%%%%%%%%%%%%%%%%%%%%%%%%%%%%%%%%%%%%%%%%%%%%%%%%%%%%
%%% ANHANG ENDE
%%%%%%%%%%%%%%%%%%%%%%%%%%%%%%%%%%%%%%%%%%%%%%%%%%%%%%%%%%%

\end{document}